\documentclass[10pt]{amsart}
\usepackage[total={6.5in,9in},centering]{geometry}
%\documentclass{article}
\usepackage{amsmath}
\usepackage{graphicx} % for png figures
\usepackage{subcaption} % for subfigures
\usepackage{xfrac} % for sfrac (slanted inline fractions)
%\usepackage{amsaddr} % addresses up front
%\usepackage{amssymb, amsmath, bm, nameref}
\pagenumbering{gobble} % no page numbers

\newcommand{\Iimg}{\mathcal{I}}
\newcommand{\Pimg}{\mathcal{P}}

\newcommand{\imgmeashat}{\pmb{\hat{x}_{i}}}
\newcommand{\imgmeas}{\pmb{x_{i}}}
\newcommand{\grndhat}{\pmb{\hat{X}}}
\newcommand{\grnd}{\pmb{X}}
\newcommand{\grnditer}{\pmb{X^k}}
\newcommand{\sensmeashat}{\pmb{\hat{p}_i}}
\newcommand{\sensmeas}{\pmb{p_i}}
\newcommand{\imgnu}{\pmb{\nu_{i}^m}}
\newcommand{\sensnu}{\pmb{\nu_i^p}}
\newcommand{\grndupdate}{\pmb{\Delta}}
\newcommand{\grndupdateiter}{\pmb{\Delta^k}}
\newcommand{\Fimgpartials}{\frac{\partial{\pmb{F_{i}}}}{\partial{\imgmeas}}}
\newcommand{\Fgrndpartials}{\frac{\partial{\pmb{F_{i}}}}{\partial{\grnd}}}
\newcommand{\Fsenspartials}{\frac{\partial{\pmb{F_{i}}}}{\partial{\sensmeas}}}
\newcommand{\btwbi}{B'_iW_iB_i}
\newcommand{\btwbii}{B'_{i+1}W_{i+1}B_{i+1}}
\newcommand{\btwbj}{B'_jW_jB_j}
\newcommand{\btwbone}{B'_1W_1B_1}
\newcommand{\btwbtwo}{B'_2W_2B_2}
\newcommand{\btwbthr}{B'_3W_3B_3}
\newcommand{\btwbfor}{B'_4W_4B_4}
\newcommand{\SET}{SET\texttrademark}

\newtheorem{theorem}{Theorem}


\begin{document}


\title[SET Enumeration]{Complete Enumeration of the Game \SET for up to 12 Cards}
\author[Settergren]{Reuben Settergren}
\email{rjs@jhu.edu}

\thanks{Thanks to Liz McMahon and Jeff Gordon of Lafayette College for helpful
  correspondence during the development of this capability, as well as for
  writing the delightful book {\em The Joy of Set} \cite{JOS}.}

\maketitle

\begin{abstract}
For the card game \SET, the exact probabilities of all possible numbers of SETs
can be determined analytically for deals of small numbers of cards, and can be
enumerated exhaustively for somewhat larger numbers. But for 12 cards (the
starting number of cards according to the rules of \SET) combinatorial explosion
has so far prohibited direct enumeration, allowing only estimation of
probabilities by simulation. In this result, massively parallel programming on a
Graphical Processing Unit (GPU) is used to enumerate all possible deals of up to
12 cards.
\end{abstract}

\section{Introduction}
In the card game \SET, each card depicts 4 attributes, and each attribute
(number, texture, color, shape) has 3 possible values, for a total of $3^4=81$
cards. A `SET' is defined as a set\footnote{Note that \SET will refer to the
  card game, and `SET' will refer to a SET in the game of \SET, while the normal
  mathematical sense (an unordered collection) will be called just a `set'.} of
3 cards for which are either all the same, or all different, in each
attribute. As such, the card game is equivalent to the finite geometry AG(4,3);
cards are analogous to points, and SETs are lines (sets of collinear
points). Another equivalent formulation, is that cards are 4-tuples from the set
$\{0,1,2\}^4$, and a SET is three cards such that their sum is $\equiv (0,0,0,0)
\pmod 3$.

The rules of \SET specify that a game begins by dealing 12 cards. As long as no
SET is present, 3 more cards are dealt. So the question naturally arises, what
is the probability that 12 cards have no SET? Or more generally, what is the
probability distribution of all possible numbers of SETs that 12 cards may
contain?

\section{Previous Results}
A complete enumeration must consider $\binom{81}{12} \approx 7\times 10^{13}$
cases. McMahon, Gordon et al note (\cite{JOS} p. ?) that, even though their
computer program can process about 100 million deals per hour, a complete
enumeration would still take 80 years. ``a faster solution would require more
clever programming, or massively parallel programming. We don't think more
cleverness is possible.''

Donald Knuth tackled the more specific problem of counting how many of the
$\binom{81}{12}$ combinations have 0 SETs in 1996, and refined it in 2001
(\cite{SETSET}). Knuth achieves tremendous speedup with ``isomorph rejection''
(i.e. cleverness). For instance, using the 4-tuple representation, the first
1-card deal is $(0,0,0,0)$. That does not contain a SET, and there are 81 total
1-card deals which do not have SETs because they are isomorphic to the first
one. Because the other 80 1-card sets are isomorphic to the first, Knuth's
program doesn't need to enumerate them.

The 2- and 3-card deals are similarly simple, but the 4-card case is more
illustrative. Knuth's enumeration sets of 4 cards goes no further than
$A=\{(0,0,0,0), (0,0,0,1), (0,0,1,0), (0,0,1,1)\}$ and $B=\{(0,0,0,0),
(0,0,0,1), (0,0,1,0), (0,1,0,0)\}$. Those sets determine isomorphism groups of
63180 and 1516320 no-SET 4-sets, respectively. Knuth's clever solution is able
to prune the 4-card enumeration beyond that point by determining that all
remaining 4-sets either is isomorphic to one of those two, or contains a SET. So
there are exactly $63180+1516320=1579500$ of all possible
$\binom{81}{4}=1663740$ 4-sets contain no SET. That leaves 1080 4-sets, these
all contain 1 SET. This result is straightforwardly obtained analytically
(\cite{JOS}, pp ??).

The analytical evaluation of 5-card deals (\cite{JOS}, pp ??) yields 22441536
5-sets with 0 SETs, 3116880 5-sets with 1 SET, and 63180 5-sets with 2
SETs. Knuth's solution also yields 3116880 0-SET 5-sets, but does not provide
the distinction between 1-SET and 2-SET sets, because the search is pruned
whenever a first SET is encountered.

Knuth's analysis of the 0-SET problem is complete, that is he proviedes counts
of 0-SET N-sets for $N\in\{1,2,\ldots,21\}$. It known (\cite{MAXCAP}) that all
sets of at least $N=21$ cards contain SETs. In particular, Knuth provides a
value of 2284535476080 SETless 12-sets, which our enumeration will match, as
well as providing the number of 12-sets which have 1 SET, 2 SETs, etc.


\section{My Stuff}


\begin{table}
  \caption{Enumeration Results}
  \centering
  \begin{tabular}{r | r r r r r r }
    SETs & 3 & 4 & 5 & 6 & 7 & 8   \\
    \hline
0  & 84240 & 1579500 & 22441536 & 247615056 & 2144076480 & 14587567020  \\
1  & 1080 & 84240 & 3116880 & 71772480 & 1137240000 & 12981215520 \\
2  &  &  & 63180 & 5068440 & 184485600 & 3996514080 \\
3  &  &  &  & 84240 & 11372400 & 573168960 \\
4  &  &  &  &  &  & 22744800 \\
5  &  &  &  &  & 42120 & 3032640 \\
6  &  &  &  &  &  &  \\
7  &  &  &  &  &  &  \\
8  &  &  &  &  &  & 10530 \\
\hline
$\binom{81}{N}$ & 85320 & 1663740 & 25621596 & 324540216 & 3477216600 & 32164253550  \\
\hline
seconds & 0.001 & 0.002 & 0.045 & 0.56 & 6.3 & 69 
  \end{tabular}

  \vskip 5mm

  \begin{tabular}{r | r r r r }
    SETs & 9 & 10 & 11 & 12 \\
    \hline
0  & 77541824880 & 318294370368 & 991227481920 & 2284535476080 \\
1  & 108956689920 & 676366666560 & 3091339021680 & 10266579666720 \\
2  & 56941354080 & 557948898000 & 3829696640640 & 18459179294400 \\
3  & 15417548640 & 253318946400 & 2704419900000 & 19278240770880 \\
4  & 1857807900 & 62354111040 & 1144078603200 & 12746054337120 \\
5  & 160729920 & 9176162112 & 306795509280 & 5650817178240 \\
6  & 11119680 & 827405280 & 49231877760 & 1649199670560 \\
7  & & 78848640 & 6382949040 & 330527433600 \\
8  & 758160 & 23882040 & 729349920 & 48500063820 \\
9  &  & 3032640 & 243622080 & 8323838640 \\
10 &  &  & 21228480 & 2091005280 \\
11 &  &  & & 160729920 \\
12 & 1170 & 84240 & 2611440 & 86346000 \\
13 &  &  & 379080 & 21340800 \\
14 &  &  &  & 3032640 \\
\hline
$\binom{81}{N}$ & 260887834350 & 1878392407320 & 12124169174520 & 70724320184700 \\
\hline
seconds & 646 & 5697 & 45298 & 324480
  \end{tabular}
\end{table}

\section{Conclusion}


\begin{thebibliography}{9} % 1-digit reference nums

\bibitem{JOS} McMahon, Liz, Gary Gordon, et al, 2016. {\em The Joy of SET: The Many
  Mathematical Dimensions of a Seemingly Simple Card Game}, Princeton University Press, Princeton NJ.

\bibitem{SETSET}Knuth, Donald E., {\sc setset} (1996) and {\sc setset-all}
  (2001), Documented programs in CWEB, from {\tt https://
      www-cs-faculty.stanford.edu/\~knuth/programs/setset.w} and {\tt
  setset-all.w}. Accessed 2019-06-06

\bibitem{ME}Settergren, Reuben, 2019. {\tt http://github.com/RubeRad/SET}

\bibitem{MAXCAP}Pellegrino, G., 1971. ``Title'', {\em Matematiche} {\bf 25}, pp
  149-157.

\end{thebibliography}
 
\end{document}
