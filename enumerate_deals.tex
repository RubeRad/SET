\documentclass[10pt]{amsart}
\usepackage[total={6.5in,9in},centering]{geometry}
%\documentclass{article}
\usepackage{amsmath}
\usepackage{graphicx} % for png figures
\usepackage{subcaption} % for subfigures
\usepackage{xfrac} % for sfrac (slanted inline fractions)
\usepackage{hyperref}
\usepackage[yyyymmdd]{datetime}
\renewcommand{\dateseparator}{-}
%\usepackage{amsaddr} % addresses up front
%\usepackage{amssymb, amsmath, bm, nameref}
\pagenumbering{gobble} % no page numbers

\newcommand{\Iimg}{\mathcal{I}}
\newcommand{\Pimg}{\mathcal{P}}

\newcommand{\imgmeashat}{\pmb{\hat{x}_{i}}}
\newcommand{\imgmeas}{\pmb{x_{i}}}
\newcommand{\grndhat}{\pmb{\hat{X}}}
\newcommand{\grnd}{\pmb{X}}
\newcommand{\grnditer}{\pmb{X^k}}
\newcommand{\sensmeashat}{\pmb{\hat{p}_i}}
\newcommand{\sensmeas}{\pmb{p_i}}
\newcommand{\imgnu}{\pmb{\nu_{i}^m}}
\newcommand{\sensnu}{\pmb{\nu_i^p}}
\newcommand{\grndupdate}{\pmb{\Delta}}
\newcommand{\grndupdateiter}{\pmb{\Delta^k}}
\newcommand{\Fimgpartials}{\frac{\partial{\pmb{F_{i}}}}{\partial{\imgmeas}}}
\newcommand{\Fgrndpartials}{\frac{\partial{\pmb{F_{i}}}}{\partial{\grnd}}}
\newcommand{\Fsenspartials}{\frac{\partial{\pmb{F_{i}}}}{\partial{\sensmeas}}}
\newcommand{\btwbi}{B'_iW_iB_i}
\newcommand{\btwbii}{B'_{i+1}W_{i+1}B_{i+1}}
\newcommand{\btwbj}{B'_jW_jB_j}
\newcommand{\btwbone}{B'_1W_1B_1}
\newcommand{\btwbtwo}{B'_2W_2B_2}
\newcommand{\btwbthr}{B'_3W_3B_3}
\newcommand{\btwbfor}{B'_4W_4B_4}
\newcommand{\SETb}{SET\texttrademark\ } % this one has a space after
\newcommand{\SET}{SET\texttrademark}  % this one has no space, so it can be
                                       % followed by a comma or period

\newtheorem{theorem}{Theorem}


\begin{document}


\title[SET Enumeration]{Complete Enumeration of the Game \SETb for up to 12 Cards}
\author[Settergren]{Reuben Settergren}
\email{rjs@jhu.edu}

\thanks{Thanks to Liz McMahon and Jeff Gordon of Lafayette College for helpful
  correspondence during the development of this capability, as well as for
  writing the delightful book {\em The Joy of Set} \cite{JOS}.}

\maketitle

\begin{abstract}
For the card game \SET, the exact probabilities of all possible numbers of SETs
can be determined analytically for deals of small numbers of cards, and can be
enumerated exhaustively for somewhat larger numbers. But for 12 cards (the
starting number of cards according to the rules of \SET) combinatorial explosion
has so far prohibited direct enumeration, allowing only estimation of
probabilities by simulation. In this result, massively parallel programming on a
Graphical Processing Unit (GPU) is used to enumerate all possible deals of up to
12 cards. The probability that an opening deal of 12 \SETb cards does not
contain a SET is exactly $2284535476080/70724320184700 \approx 3.2302\%$.
\end{abstract}

\section{Introduction}
In the card game \SET, each card depicts 4 attributes, and each attribute
(number, texture, color, shape) has 3 possible values, for a total of $3^4=81$
cards. A `SET' is defined as a set\footnote{Note that \SETb will refer to the
  card game, and `SET' will refer to a SET in the game of \SET, while the normal
  mathematical sense (an unordered collection) will be called just a `set'.} of
3 cards for which are either all the same, or all different, in each
attribute. As such, the card game is equivalent to the finite geometry AG(4,3);
cards are analogous to points, and SETs are lines (sets of collinear
points).

Extending the geometric analogy, just as 3 non-collinear points determine a
plane (and all the lines between points in the plane lie in the plane), a
'plane' in \SETb can be built from 3 cards which do not form a SET, by adding
cards which form SETs with 2 previous cards. Closure will be reached at 9 cards,
at which point for any two cards, the third card which completes a SET with them
will also be present. The number of SETs in a plane is 12. See \cite{JOS} for
extensive discussion of planes, hyperplanes, and the relationship between \SETb
and affine geometry in general.

Another equivalent formulation, is that cards are 4-tuples from the set
$\{0,1,2\}^4$, and a SET is three cards such that their sum is $\equiv (0,0,0,0)
\pmod 3$. Even more briefly, the 4-tuples can be represented by 4-digit numbers
in base 3; the 81-cards in the \SETb deck are represented by ternary numbers
0000, 0001, 0002, 0010, \ldots 2222. A fundamentally important fact about SETs
is that, for any two distinct cards, there is a unique third card which
completes a SET with them. To find it, set each attribute/digit of the third
card appropriately so that each digit sum is $\equiv 0 \pmod 3$.

The rules of \SETb specify that a game begins by dealing 12 cards. As long as no
SET is present, 3 more cards are dealt. So the question naturally arises, what
is the probability that 12 cards have no SET? Or more generally, what is the
probability distribution of all possible numbers of SETs that 12 cards may
contain?

\section{Previous Results}
A complete enumeration must consider $\binom{81}{12} \approx 7\times 10^{13}$
cases. McMahon, Gordon et al note (\cite{JOS} p.~259) that, even though their
computer program can process about 100 million deals per hour, a complete
enumeration would still take 80 years. Enumeration, they note, ``might be
feasible with streamlined code and faster machines or parallel processing.''

Donald Knuth tackled the more specific problem of counting how many deals have 0
SETs. Knuth achieves tremendous speedup over exhaustive enumeration by using
``isomorph rejection'' (i.e. `streamlined code'). Knuth improved on his first
solution \cite{SETSET} by expanding the size of his isomorphism groups
\cite{SETSET-ALL}, thus requiring the consideration of fewer specific cases.

For instance, using the 4-digit ternary representation, the first 1-card deal is
simply $\{0000\}$. That does not contain a SET, and there are 81 total 1-card
deals which do not have SETs because they are isomorphic to the first
one. Because the other 80 1-card sets are isomorphic to the first, Knuth's
program doesn't need to enumerate them.

The 2- and 3-card deals are similarly simple, but the 4-card case is more
illustrative. Knuth's enumeration of sets of 4 cards goes no further than
$\{0000, 0001, 0010, 0011\}$ and $\{0000, 0001, 0010, 0100\}$.\footnote{Note
  card 0002 is omitted in these sets, because it forms a set with 0000 and
  0001.}  Those sets determine isomorphism groups of 63180 and 1516320 no-SET
4-sets, respectively. Knuth is able to prune the 4-card enumeration beyond that
point by determining that all remaining 4-sets either is isomorphic to one of
those two, or contains a SET. So there are exactly $63180+1516320=1579500$ of
all possible $\binom{81}{4}=1663740$ 4-sets contain no SET. That leaves 1080
4-sets -- it is straightforward to analytically determine that all 1080 of them
contain exactly 1 SET (\cite{JOS}, p.~55).

The analytical evaluation of 5-card deals (\cite{JOS}, pp.~56-9) yields 22441536
5-sets with 0 SETs, 3116880 5-sets with 1 SET, and 63180 5-sets with 2 SETs (see
also Table \ref{THE_TABLE}). Knuth's solution also yields 3116880 0-SET 5-sets,
but does not provide the distinction between 1-SET and 2-SET counts, because the
search is pruned whenever a first SET is encountered.

Knuth's analysis of the 0-SET problem is complete, in that he provides counts of
0-SET $k$-sets for $k$ up to 21. It known \cite{MAXCAP} that all sets of at
least $k=21$ cards contain SETs (a set of 20 cards with no SET corresponds to
what is called a 'maximal cap' in affine geometry). In particular, Knuth
provides a value of 2284535476080 SETless 12-sets, which our enumeration will
match, as well as providing the number of 12-sets which have 1 SET, 2 SETs, etc.


\section{Implementation}
This paper picks up the implicit challenge of McMahon and Gordon to code the
12-card enumeration with streamlined code and parallel processing (although not
necessarily faster machines). Source code is available on github \cite{ME}. The
code can be run serially, or in parallel using threads or OpenCL \cite{OPENCL}
kernels. Note that OpenCL can be configured so that its kernels can be
parallelized onto either CPU or GPU resources.

For each set of $k$ cards, all $\binom{k}{3}$ triples need to be checked for
whether they constitute a SET. Knuth notes, ``We will frequently need to find
the third card of a SET, given any two distinct cards $x$ and $y$, so we store
the answers in a precomputed table.''  This is an immense computational savings
compared to 4 sums and 4 modulo 3's for each SET determination, so that practice
was adopted here as well.

Sets of $k$ cards ('$k$-sets) are enumerated according to a combinatorial
numbering system. In the first iteration of the enumeration program, each
combination number from 0 to $\binom{81}{k}-1$ was fully `unranked' to a
$k$-combination using the algorithm presented in \cite{WIKI}. In order to unrank
an combinatorial number $N$ into $k$ specific cards, the algorithm requires, for
each of the $k$ cards, a number of comparisons to $\binom{n}{k}$, for $n\le 81,
k\le 12$. To avoid $O(k)$ computation each for each binomial coefficient, those
values were also precomputed and stored in a lookup table. The number of lookups
of $\binom{n}{k}$ required to unrank a particular number $N$ varies. The fixed
number of 81 cards establishes a generous upper bound on the possible number of
lookups per card being unranked, and the minimum number of steps is 1. Since the
number of steps for each of the $k$ cards is betwen 1 and 81, the complexity of
this unranking algorithm is $O(k)$. The code was temporarily instrumented to
count steps, and it was empirically determined that the constant for this linear
complexity (the average number of $\binom{n}{k}$ lookups for unranking a
$k$-set) is $C\approx 45$. (This could be improved by switching the incremental
searching of the binomial table to binary searching, but the need for
non-branching code (see immediately below) obviates this computational approach
anyways.)

Later, a combination-incrementing algorithm was implemented, with no branching
(GPU programs are most efficient if there are no {\tt if} or {\tt while}
statements, because simultaneous kernels execute the same line of code together,
and kernels that take different paths through the code have to wait for each
other). Instead of fully-unranking each combination from its enumeral number, we
reuse the previous combination by incrementing it to the next combination in the
series. Because of the avoidance of branching, the combination-increment
algorithm takes exactly the same number of steps to increment any
combination. The algorithm executes 5 lines of code for each of the $k$ cards in
the combination, so the incrementing algorithm is also $O(k)$. The algorithm is
not 9x faster, however, since the 5 lines are more complex than the simple steps
taken in the above unranking algorithm. Empirically, the speedup from switching
to full unranking to incrementation was 2-3x (even serially).

The goal for this paper was to enumerate 12-card deals, which requires
enumeration of combinations numbered by integers as large as $\binom{81}{12}-1
\approx 7\times 10^{13}$. For this purpose, an unsigned 64-bit integer
suffices. Fortunately, the latest advances in GPU technology have recently
introduced native capabilty for 64-bit integer types. Also fortuitous is the
fact that, since $2^{64} > \binom{81}{21}$, 64-bit unsigned integers suffice all
the way up to deals of 21 cards, which is the same extent as Knuth's programs
analyzed (since 20 is the size of the maximal cap -- there are configurations of
20 cards without SETs, and $k=21$ is the minimal deal size for which the number
of 0-SET deals is 0).

This problem is `embarassingly parallel': counting the number of sets in any
combination of cards is completely independent of any other. What's more, the
input is merely a number (or two numbers defining a range), and the output a
small array of accumulators, counting how many $k$-sets in the range have 0
SETs, 1 SET, \ldots 14 SETs (it is known that 14 is the maximum number of SETs
that can occur in 12 cards \cite{VINCI}). Each kernel can have its own small
array of counters, to avoid write contention, and the counter arrays can be then
quickly summed.

\section{Results}
The exhaustive enumeration program was run for $3<=k<=12$. The results are
presented in Table \ref{THE_TABLE}. Runtimes are for GPU processing on an
NVIDIA Quadro P4000. Non-rigorous trial and error determined that a decent
setting for the GPU processing was to queue 5000 kernels at a time, each tasked
with enumerating a batch of 10000 $k$-combinations.



The 12-set enumeration took 90.1h=3.75d. The total time for $k=3\ldots 12$ was
104.5h=4.35d. (Because of combinatorial explosion, the 12-card enumeration
consumed the lion's share of processing.) The 12-card processing rate was 218M
deals/second, and the overall rate was 226M/second (slightly higher because each
$k<12$-combination required checking $\binom{k<12}{3}$ triples,
etc). Single-threaded on a core i7 2.8GHz CPU (?), the 12-card enumeration ran
at 842K/s (projected exhaustive enumeration time 972d=2.66y). Thus GPU
processing yielded a 218M/842k$\approx$256x speedup (at least comparing these specific
processors). However, the Quadro P4000 has 1792 cores, so it seems possible that
the GPU processing could be further optimized to obtain a 12-card enumeration
time well under 24 hours. Enumeration for $k=13$ or maybe 14 may become
worthwhile, but combinatorial explosion quickly wins again, as runtimes increase
about 7x for each larger $k$. Perhaps Knuth's isomorphism rejection technique
could be extended to counting not just 0-SET deals, making a complete
enumeration of up to $k=81$ possible.

\begin{table}
  \caption{Enumeration Results}\label{THE_TABLE}
  \centering
  \begin{tabular}{r | r r r r r r }
    & \multicolumn{6}{c}{Number of cards} \\
    \hline\hline
    SETs & 3 & 4 & 5 & 6 & 7 & 8   \\
    \hline
0  & 84240 & 1579500 & 22441536 & 247615056 & 2144076480 & 14587567020  \\
1  & 1080 & 84240 & 3116880 & 71772480 & 1137240000 & 12981215520 \\
2  &  &  & 63180 & 5068440 & 184485600 & 3996514080 \\
3  &  &  &  & 84240 & 11372400 & 573168960 \\
4  &  &  &  &  &  & 22744800 \\
5  &  &  &  &  & 42120 & 3032640 \\
6  &  &  &  &  &  &  \\
7  &  &  &  &  &  &  \\
8  &  &  &  &  &  & 10530 \\
\hline
$\binom{81}{N}$ & 85320 & 1663740 & 25621596 & 324540216 & 3477216600 & 32164253550  \\
\hline
seconds & 0.001 & 0.002 & 0.045 & 0.56 & 6.3 & 69 
  \end{tabular}

  \vskip 5mm

  \begin{tabular}{r | r r r r }
    SETs & 9 & 10 & 11 & 12 \\
    \hline
0  & 77541824880 & 318294370368 & 991227481920 & 2284535476080 \\
1  & 108956689920 & 676366666560 & 3091339021680 & 10266579666720 \\
2  & 56941354080 & 557948898000 & 3829696640640 & 18459179294400 \\
3  & 15417548640 & 253318946400 & 2704419900000 & 19278240770880 \\
4  & 1857807900 & 62354111040 & 1144078603200 & 12746054337120 \\
5  & 160729920 & 9176162112 & 306795509280 & 5650817178240 \\
6  & 11119680 & 827405280 & 49231877760 & 1649199670560 \\
7  & & 78848640 & 6382949040 & 330527433600 \\
8  & 758160 & 23882040 & 729349920 & 48500063820 \\
9  &  & 3032640 & 243622080 & 8323838640 \\
10 &  &  & 21228480 & 2091005280 \\
11 &  &  & & 160729920 \\
12 & 1170 & 84240 & 2611440 & 86346000 \\
13 &  &  & 379080 & 21340800 \\
14 &  &  &  & 3032640 \\
\hline
$\binom{81}{N}$ & 260887834350 & 1878392407320 & 12124169174520 & 70724320184700 \\
\hline
seconds & 646 & 5697 & 45298 & 324480
  \end{tabular}
\end{table}


A number of interesting features arise from the full pattern of counts. The
value of 84240 shows up 4 times, 3032640 three times, and 160729920 twice. This
invites questions such as, is there a meaningful bijection between the 9-sets
with 5 SETs, and the 12-sets with 11 SETs? Or is it just a coincidence from
reuse of the same prime factors?


There are a number of gaps in the table indicating impossible numbers of SETs
for some set sizes. The first gap shows that 7 cards cannot have 1, 2, 3, or 5,
but not exactly 4 SETs. Although there are 84240 6-sets with 3 SETs, there is no
way to add a $7^{th}$ card and complete only one more SET.\footnote{And yet
  there are 42120 7-sets with 5 SETs -- do they correspond with half of the
  84240 3-SET 6-sets for a meaningful reason? Or do some of the result from
  adding a card to a 6-set with fewer than 3 SETs, and completing more than 2
  new SETs?}

The gaps in the 9-set enumeration are quite interesting. The 1170 9-sets with 12
SETs are all possible planes (this matches the analytical result obtained in
\cite{JOS}, p.~48). It is possible for 9 cards to have 8 SETs, as explained by
Liz McMahon:\footnote{Co-author of \cite{JOS}, personal correspondence}
\begin{quote}
If you take a plane [12 SETs] and remove one card, what’s left has 8 SETs.  You
can then add another point that doesn’t complete any more SETs, and that gives
you the 8 SETs in 9 cards.
\end{quote}
Between 8 and 12 SETs though, no intermediate number of SETs is possible for 9
cards.

Every occurrence of a gap for $k$ cards is followed by a gap that is smaller by
one in $k+1$ cards. 


That gap carries forward (diminishing) in that 10- and 11-sets can also
have 12 SETs (planes plus 1 or 2 cards that complete no new SET), but 10-sets
cannot have 10 or 11 SETs, and 11-sets cannot have 11 SETs.



\section{Conclusion}
Exhaustive enumeration of all sets of up to 12 \SETb cards is accomplished using
highly-parallelized GPU programming, in a little over 100 hours. The
implementation is described, and the results discussed.


\begin{thebibliography}{9} % 1-digit reference nums


\bibitem{SETSET}Knuth, Donald E., {\sc setset} (1996), Documented programs in
  CWEB, from
  \url{https://www-cs-faculty.stanford.edu/~knuth/programs/setset.w}. Accessed
  \today.

\bibitem{SETSET-ALL}Knuth, Donald E., {\sc setset-all} (2001), Documented
  programs in CWEB, from
  \url{https://www-cs-faculty.stanford.edu/~knuth/programs/setset-all.w}. Accessed
  \today.

\bibitem{JOS} McMahon, Liz, Gary Gordon, et al, 2016. {\em The Joy of SET: The
  Many Mathematical Dimensions of a Seemingly Simple Card Game}, Princeton
  University Press, Princeton NJ.

\bibitem{OPENCL} Open Computing Language (OpenCL), from {\tt
  khronos.org/opencl}, accessed \today.

\bibitem{MAXCAP}Pellegrino, G., 1971. ``Sul massimo ordine delle calotte in
  $S_{4,3}$'' [{\em The maximal order of the shperical cap in $S_{4,3}$}], {\em
  Matematiche} {\bf 25}, pp 149-157.

\bibitem{ME}Settergren, Reuben, 2019. \url{https://github.com/RubeRad/SET}
  
\bibitem{VINCI}Vinci, Jim, 2009. ``The maximum number of sets for $n$ cards and
  the total number of internal sets for all partitions of the deck,'' on the
  \SETb game website at
  \url{https://www.setgame.com/sites/default/files/teacherscorner/SETPROOF.pdf},
  accessed \today.

\bibitem{WIKI} Wikipedia article ``Combinatorial Number System,'' \url{https://en.wikipedia.org/wiki/Combinatorial_number_system#Finding_the_k-combination_for_a_given_number},
  accessed \today.

\end{thebibliography}
 
\end{document}
